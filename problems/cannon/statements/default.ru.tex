% Statements
Петя разведчик, и ему удалось добыть для своего государства
важную информацию о расположении вражеского стратегического
объекта. Чтобы его поразить, можно воспользоваться
пушкой, которая находится на вражеской же территории, однако
Петя сумел взломать только первый уровень защиты боевой
системы, поэтому он может контролировать момент выстрела
из пушки, но абсолютно не знает, в какую сторону эта пушка
смотрит и на какую высоту нацелена. Из некоторых косвенных
источников известно, что пушка смотрит по сторонам равновероятно
и распределение вероятности попадания в точку, удаленную
от пушки на расстояние $x$, имеет экпоненциальный закон
$exp(-x)$. Так как Пете нужно выполнить задание по нейтрализации
стратегического объекта как можно раньше, он попросил Вас
оценить вероятность попадания в объект из пушки, чтобы
в случае, когда вероятность мала, остаться нераскрытым
и использовать более надежный способ выполнения миссии.

\InputFile
% Input file description
В первой строке содержится число $n, 3 \le n \le 300$ ~--- число узловых точек
вражеского объекта, который представляет собой невырожденный выпуклый многоугольник.
В данных содержатся шумы, и некоторые точки не нужны для определения формы
объекта, однако в каждой его вершине есть хотя бы одна узловая точка.
Далее в отдельных строках даны по два целых числа
$x_i\,\,y_i, -500 \le x_i, y_i \le 500$ ~--- координаты узлов.
Пушка находится в точке $(0,0)$.
\OutputFile
% Output file description
Выведите единственное число с не менее чем 6 знаками после десятичной точки
~--- вероятность попадания во вражеский объект из пушки.

