% Statements
Число $x$ называется корнем $n$-ой степени числа $a$ по модулю $p$
тогда и только тогда, когда $x^n = a \,(mod \,\,p)$.
Напишите программу, которая находит все корни степени $n$ из числа $a$
по модулю $p$.

\InputFile
% Input file description
В первой строке находится одно число $t$ ($1 \le t \le 10000$) ~---
Каждая следующая строка представляет собой отдельный тест,
который содержит целые числа $a$, $n$ и $p$,
$1 \le a, p \le 10^6, 1 \le n \le 10$,
$p$ ~--- простое, $a$ и $p$ ~--- взаимно простые.
\OutputFile
% Output file description
Для каждого теста выведите все корни степени $n$ из $a$ в диапазоне $[0, p-1]$
в возрастающем порядке в одной строке, разделяя одним пробелом.
Если для текущего теста корней не существует,
выведите в отдельной строке сообщение <<No root>>.
