Пусть дано $a$, $p$ и $n$, требуется найти все
решения уравнения $x^n = a\,(mod \, p)$.
Так как $p$ ~--- простое, то можно отыскать
порождающий элемент поля $\mathbb{Z}_p$.
Считается, что порождающий элемент можно искать
перебором натуральных чисел, так как он относительно мал.
Существует асимптотическая оценка Шупа величины минимального
порождающего элемента $O(\log^6(p)$, однако она верна по некоторым
источникам в предположении истинности гипотезы Римана.
На практике же перебор и проверка, что число является
порождающим элементом поля является эффективным алгоритмом
для решения ряда алгоритмических теоретико-числовых задач.\\

Пусть мы хотим проверить, является ли число $x$
порождающим элементом. Так как $p$ ~--- простое,
то для него и $x$ верно тождество $x^{p-1} = 1\,(mod \,p)$.
Чтобы $x$ был искомым необходимо и достаточно, чтобы
не существовало показателя $1 \le d < p - 1$, для которого
$x^d = 1\,(mod \,p)$. $d$ должен быть делителем $p-1$,
в противном случае должно было быть $x=1\,(mod \,p)$.
Рассмотрим разложение на простые множители
$p-1=p_1^{\alpha_1}\cdot \ldots \cdotp_k^{\alpha_k}$.
Так как $d$ делитель $p-1$ строго меньший его, то
он делит какой-то элемент из множества $K = \{(p-1)/p_i|i=\overline{1,k}\}$.
Если $x^d=1\,(mod \,p)$, то подавно $x^{dl}=1\,(mod\,p), l\in\mathbb{N}$.
Поэтому достаточно проверить, что $d^x \neq 1\,(mod \,p), x \in K$.
Если возводить в степень бинарно, что поиск производящего
элемента можно осуществить за $O(ans\log^2(p)$.\\

Пусть $e$ ~--- производящий элемент, тогда
задачу поиска корней уравнения сводится к задаче поиска дискретного
логарифма $(e^n)^x = a\,(mod \,p)$.\\

Как найти дискретный логарифм $b^x=a\,(mod \, p)$?
Предаставим $x$ как
$x = m * s - l, l = \overline{1,m}, s = \overline{1,\lceil \frac{p}{m} \rceil}$.
Тогда уравнение перепишется так:
$(b^m)^s = ab^l\,(mod \, p)$. Если выбрать $m \equiv \sqrt{p}$, то
можно найти решение с помощью метода <<meet in the middle>> за
$O(\sqrt(p)\log(p)$.\\

Пусть $x_0$ ~--- значение дискретного логарифма, и
$e^{x_0n}=a\,(mod \,p)$. В силу малой теоремы
Ферма также имеет место $e^{x_0n+t(p-1)}=a\,(mod \,p)$.
Для всех $0 \le t < n$, если $n | t(p-1)$, то $x0+t(p-1)/n$
будет корнем $n$-ой степени из $a$. Нетрудно увидеть, что так
находятся все корни.
