% Statements
Число $x$ называется квадратным корнем числа $a$ по модулю $n$
тогда и только тогда, когда $x^2 = a (mod \,\,n)$.
Напишите программу, которая находит все квадратные корни числа $a$
по модулю $n$.

\InputFile
% Input file description
В первой строке находится одно число $t$ ($1 \le t \le 100000$) ~---
Каждая следующая строка представляет собой отдельный тест,
который содержит целые числа $a$ и $n$, $1 \le a, n \le 32767$,
$n$ ~--- простое, $a$ и $n$ ~--- взаимно простые.
\OutputFile
% Output file description
Для каждого теста выведите все квадратные корни $a$ в диапазоне $[0, n-1]$
в возрастающем порядке в одной строке, разделяя одним пробелом.
Если для текущего теста корней не существует,
выведите в отдельной строке сообщение <<No root>>.
