Пусть есть рекуррентная последовательность
$a_n = c_{k-1}a_{n-1} + \ldots + c_0a_{n-k}$.
Попробуем найти решение в виде $a_n = x^n$. После подстановки,
исключая тривиальное решение,
получаем уравнение $x^k - c_{k-1}x^{k-1} - \ldots - c_0 = 0$.
У данного уравнения, согласно общей теореме алгебры,
ровно $k$ корней c учетом их кратности над полем комплексных чисел,
а при целочисленности коэффициентов и над полем $Z_{p}$, $p$ ~---
простое не меньшее $k$.
Пусть $x_1, x_2, \ldots, x_l$ ~--- корни этого полинома
над соответствующим полем, а $m_1, \ldots, m_l$ ~--- их
кратности, $\sum_{i=1}^{l}m_i = k$. Очевидно, что
$a_n = n^{t_i}x_i^n, 0 \le t_i < m_i, i=\overline{1, l}$
является решением рекуррентности. Понятно, что линейная
комбинация этих решений тоже решение. Имея $k$ неизвестных
коэффициентов при независимых решениях и $k$ начальных
условий $a_0 = A_0, \ldots, a_{k-1}=A_{k-1}$, можно найти решение
конкретной рекуррентности.\\

Заметим, что ассоциированный с рекуррентным соотношением
$$
a_n = \binom{k}{1} a_{n-1} - \binom{k}{2} a_{n-2} + \cdots
+ (-1)^{k+1}\binom{k}{k} a_{n-k}
$$
характеристический полином сворачивается в $(x - 1)^k$.
Таким образом, общее решение есть
$a_n = c_0 + c_1 n + \ldots + c_{k-1}n^{k-1}$, и подставляя
его в начальные условия, получаем систему линейных
алгебраических уравнений над конечным полем характеристики $P$.
Его можно решить методом Гаусса. Сложность алгоритма $k^3$.

Можно попробовать, на всякий случай, решить задачу бинарным возведением
матрицы перехода от $(a_{n-k}, \ldots, a_{n-1})$ к
$(a_{n-k+1},\ldots,a_{n})$ в степень, это будет за $k^3\log(n)$,
по идее проходить не должно по времени.
