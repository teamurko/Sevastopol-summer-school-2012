% Statements
Вася подготавливает очередной контест для весенних
Всеберляндских сборов по программированию. Так как
он любитель веселых компаний, да и сама погода
намекает, то на подготовку у него осталось совсем мало времени.
Он хочет использовать его наиболее эффективно. Как одна из задач
перед ним стоит непростая задача как можно быстрее попасть
из одной директории в другую, причем он так привык к своему
любимому файловому менеджеру <<Near Commandir>>, что не может
использовать другой софт, потому что это бы заняло у него
гораздо больше времени. Этот менеджер умеет делать
несколько простых операций: переместиться в списке файлов и поддиректорий
на одну позицию вверх, переместиться вниз, войти в поддиректорию,
которая выделена в списке в данный момент, причем если это родительская
директория, мы поднимаемся на уровень выше в дереве каталогов.
Каждая из этих простых операций занимает у Васи 1 секунду.
Для ускорения Вася также может применить операцию изменения порядка файлов
и поддиректорий в текущей директории, на эту операцию Вася затрачивает $t$
секунд. Файлы в директории можно сортировать по имени,
по размеру и времени последнего изменения. В двух последних
случаях файлы с одинаковым размером и временем сортируются по имени.
Имена файлов и каталогов в рамках одной директории уникальны.
Теперь он хочет узнать, сколько времени ему потребуется,
чтобы перейти на нужный ему файл, если известно, где он сейчас
находится в дереве файловой системы. Изначально файлы в текущей директории
отсортированы по имени, при входе в новую директорию файлы
также сортируются по имени без дополнительных временных затрат
со стороны Васи.

\InputFile
% Input file description
Файловая система задается деревом с выделенным корнем. В первой строке
содержится $n \,\, t, 1 \le n \le 100000, 1 \le t \le 10$
~--- размер дерева и время изменения порядка файлов. Далее в следующих
$n$ строках содержится по $4$ числа
$p_i \,\, name_i \,\, fsize_i \,\, date_i, 1\le i \le n$.
$p_i$ ~--- номер вершины предка, если это $-1$, эта единственная вершина
является корнем, $name_i$ ~--- имя файла, непустая строка,
состоящая из не более $10$
строчных латинских букв, $fsize_i$ ~--- размер файла,
$0 \le fsize_i \le 10000$.
$date_i$ ~--- время последнего изменения файла, $0 \le date_i \le 10000$.
В последней строке содержатся два числа $0 \le s, f < n$ ~---
номер вершины, на которой мы сейчас стоим, и номер вершины, в которую
требуется попасть. Вершины нумеруются том порядке,
в котором они подаются на вход.
\OutputFile
% Output file description
Выведите одно целое число, минимальное количество секунд.
