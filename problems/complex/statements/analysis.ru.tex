Как посчитать сложность числа
$n = p_1^{\alpha_1} p_2^{\alpha_2} \cdots p_k^{\alpha_k}$?
Разложение на простые множители делителя числа $n$
представляет выражение вида $p_1^{\beta_1}p_2^{\beta_2}\cdots p_k^{\beta_k}$,
где $0 \le \beta_i \le \alpha_i$, В силу единственности представления
целого числа в канонической форме каждый набор $(\beta_1, \ldots, \beta_k)$
порождает уникальные делители $n$,
и эти наборы исчерпывают все делители числа $n$.
Таким образом, сложность числа $n$ есть $\prod_{i=1}^{k}(1+\alpha_i)$.
Ясно, что более сложное число не превосходящее некоторого $n$
будет иметь меньшие простые множители.
Различных наименьших простых множителей можно взять
не более $15$-ти при данных ограничениях, и поэтому
хорошо и быстро работает перебор делителей с отсечениями.
