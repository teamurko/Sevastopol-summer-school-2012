Можно провести эксперимент и увидеть, что
разность между последовательными простыми числами,
не превосходящими $10^9$, не превосходит $282$,
достигается на последовательных простых $436273009$
и $436273291$. Для каждого составного числа
между последовательными простыми найдем
его простые делители. Построим двудольный
граф, в первой доле ~--- составные числа $N$,
во второй ~--- простые делители $P$. Ребро из вершины
$n \in N$ в вершину $p \in P$ проведем тогда и только тогда,
когда $p | n$. Найдем максимальное паросочетание.
