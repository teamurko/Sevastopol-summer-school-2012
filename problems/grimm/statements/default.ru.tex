% Statements
Гипотеза Гримма в теории чисел утверждает, что
каждому элементу непрерывной последовательности составных чисел
можно сопоставить уникальный простой делитель. Вам требуется
построить это соответствие.
\InputFile
% Input file description
На вход подаются два последовательных простых числа $p_1$ и $p_2$,
разделенных одним пробелом, гарантируется, что все числа между $p_1$
и $p_2$ составные ($3 \le p_1 < p_2 \le 10^9$).
\OutputFile
% Output file description
В единственной строке выведите $p_2 - p_1 - 1$ попарно различных
простых чисел, разделяя их одним пробелом, делители соответственно
$p_1 + 1, \ldots, p_2 - 1$.
